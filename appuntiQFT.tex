\documentclass[10pt,a4paper]{article}
\usepackage[utf8]{inputenc}
\usepackage[italian]{babel}
\usepackage{amsmath}
\usepackage{amsfonts}
\usepackage{caption}
\usepackage{subcaption}
\usepackage{amssymb}
\usepackage{graphicx}
\usepackage[left=2cm,right=2cm,top=2cm,bottom=2cm]{geometry}
\usepackage{svg}
\usepackage{wrapfig}
\usepackage{amsthm}     %teoremi, definizioni, ...
\usepackage{physics}    %notazione braket
\usepackage{parskip}    %skip tra paragrafi e sostituisce // 
\usepackage{bbold}      %identità
\usepackage{slashed}    %operatori di Dirac
\usepackage{hyperref}   %references

\hypersetup{
    colorlinks = true,
    linkcolor = blue,
    urlcolor = red,
    pdftitle = QFT notes by VG 
}

\newtheorem{thm}{Teo}[section]
\newtheorem{prop}{Prop}[section]
\newtheorem*{idea}{Idea}
\newtheorem{lemma}{Lemma}[section]
\newtheorem{cor}{Cor}[section]

\theoremstyle{definition}
\newtheorem{definition}{Def}[section]
\newtheorem{observation}{Oss}[section]
\newtheorem{example}{Ex}[section]

%Comandi personalizzati
\newcommand{\lagr}{\mathcal{L}} %Lagrangiana
\newcommand{\id}{\mathbb{1}}


\title{Appunti}
\author{VG}
\date{December 2022}



\begin{document}
\maketitle
\tableofcontents
\section{Funzionale generatore}
Partiamo dal caso semplice di una teoria di campo scalare neutro auto-interagente e definiamo le quantità che ci saranno utili in generale.

Consideriamo allora
\[
\lagr = \frac12 \partial_\mu\phi\partial^\mu\phi - \frac12 m^2 \phi^2 - V(\phi) = \lagr_0 - V(\phi)
\]
Si ha che la funzione di correlazione ad n punti è
\[
G_n (x_1, \dots, x_n) \equiv \ev**{T\{\hat{\phi}(x_1)\dots\hat{\phi}(x_n)\}}{\Omega} = \frac{\int\,D\phi\,e^{i \int \,d^4x\lagr(x)}\phi(x_1)\dots\phi(x_n)}{\int\,D\phi\,e^{i \int \,d^4x\lagr(x)}}
\]
\begin{definition}[Funzionale generatore]
    E' conveniente definire il funzionale generatore 
    \[
    Z[J] = \int D\phi\, e^{i \int d^4x\, (\lagr(x) + J(x)\phi(x))}    
    \]
\end{definition}
da cui risulta che 
\[
G_n(x_1, \dots, x_n) = \frac{(-i)^n}{Z[0]} \fdv{J(x_1)}\dots\fdv{J(x_n)}\eval{Z[J]}_{J = 0}    
\]
Possiamo poi sviluppare il funzionale generatore in potenze dell'interazione $V(\phi)$
\begin{align*}
    Z[J] &= \int D\phi\,e^{-i\int d^4x\,V(\phi)}e^{i \int d^4x \, (\lagr_0(x) + J(x)\phi(x))} \\
    &= e^{-i\int d^4x\,V\left(-i \fdv{J(x)}\right)} Z_0[J] \\
    &= \sum_{n = 0}^{\infty}\frac{(-i)^n}{n!}\left(\int d^4x\,V\left(-i\fdv{J(x)}\right)\right)^n Z_0[J]
\end{align*}
dove si è usato che
\[
\fdv{Z[J]}{J(x)} = \phi(x) Z[J]    
\]
$Z_0[J]$ è poi il funzionale generatore della teoria libera, che essendo un integrale di tipo gaussiano può essere facilmente calcolato. Possiamo infatti integrare per parti l'esponente ed ottenere
\begin{align*}
    Z_0[J] &= \int D\phi\,\exp\left(-\frac{i}{2}\int d^4x\,\phi(x)(\partial_\mu\partial^\mu + m^2)\phi(x) + i\int d^4x\, J(x)\phi(x)\right)  \\
    &= \int D\phi\, \exp\left( -\frac{i}{2} d^4x d^4y\, \phi(x)K(x, y)\phi(y) + i\int d^4x J(x)\phi(x) \right)
\end{align*}
con $K(x, y) \equiv \delta^4(x-y)(\square_x + m^2)$. Usando ora i risultati per gli integrali gaussiani otteniamo che
\[
Z_0[J] = \frac{1}{\sqrt{\det\left(\frac{A}{2\pi}\right)}}e^{\frac{i}{2}J^t K^{-1}J}    
\]
\begin{example}
    Calcoliamo il propagatore per la teoria di un campo scalare neutro libero
    \[
    G_2(x, y) \equiv \ev**{T\{\phi(x)\phi(y)\}}{0} = \frac{(-i)^2}{Z_0[J]}\fdv{J(x)}\fdv{J(y)}\eval{Z[J]}_{J=0}    
    \]
    Calcoliamo ora la derivata in notazione discreta \(J(x) \to J_i\)
    \[
    Z_0[J] = Z_0[0]e^{\frac{i}{2}J_m(K^{-1})_{mn}J_n}    
    \]
    \[
        \Rightarrow \pdv{Z_0[J]}{J_i} = \frac12  \left((iK^{-1})_{in}J_n + J_m(iK^{-1})_{mi} \right)Z_0[J] = (iK^{-1})_{in}J_nZ_0[J]
    \]
    dove abbiamo usato il fatto che se $K$ è simmetrica allora anche $K^{-1}$ lo è. Quindi
    \[
    G_2(x, y) = -i K^{-1}(x, y)    
    \]
    con $K^{-1}$ definito nel modo seguente
    \[
    (\square_x + m^2)K^{-1}(x,y) = \delta^4(x - y)    
    \]
    La seguente equazione si può risolvere passando in trasformata di Fourier ed usando la prescrizione $i\epsilon$. Otteniamo così che
    \[
    \tilde{K}^{-1}(k) = \frac{1}{m^2 - k^2 - i\epsilon}    
    \]
\end{example}
\section{Path integral per campi fermionici}
\subsection{Algebra di Grassmann}
Delle variabili $\eta_1,\dots, \eta_N$ generano un'algebra di Grassmann se
\[
\{\eta_i, \eta_j\} = 0\quad \forall i, j    
\]
In particolare si osserva che $\eta_i^2 = 0$.
\subsubsection{Regole di integrazione per variabili di Grassmann}
\begin{prop}
    \[
    \int d\eta_i = 0 \qquad \int d\eta_i\, \eta_i = 1    
    \]
    con
    \[
       \{d\eta_i, d\eta_j\} = \{d\eta_i, \eta_j\} = 0 \qquad \forall i, j 
    \]
\end{prop}
Il più generale elemento $f(\eta)$ di un'algebra di Grassmann è del tipo
\[
f(\eta) = f_0 + \eta_i f_i + \eta_i \eta_j f_{ij} + \dots + \eta_N \dots \eta_1 f_{1 \dots N }    
\]
Allora
\[
\int \prod_i d\eta_i \, f(\eta) = f_{1 \dots N}    
\]
\begin{prop}
    L'integrazione così definita gode dell'invarianza per traslazioni
    \[
    \int \prod_i d\eta_i \, f(\eta + a) = \int \prod_i d\eta_i'\, f(\eta') = f_{1 \dots N}     
    \]
\end{prop}

\begin{example}[Integrale gaussiano fermionico]
    Consideriamo un set di variabili di Grassmann del tipo
    \[
    \{\eta_1, \dots, \eta_N, \eta_{N + 1}, \dots, \eta_{2N}\} = \{\psi_1, \dots, \psi_N, \bar{\psi}_1, \dots, \bar{\psi}_N\}    
    \]
    E vogliamo calcolare
    \[
    I(A) \equiv \int [d\bar{\psi}d\psi]\, e^{-\bar{\psi}_i A_{ij}\psi_j}     
    \]
    dove 
    \[
        [d\bar{\psi}d\psi] = \prod_i d\bar{\psi}_i d\psi_i   
    \]
    e A invertibile. Si trova allora che
    \[
    I(A) = \det A    
    \]
\end{example}

\begin{example}[Funzionale generatore fermionico]
    Consideriamo il seguente funzionale generatore 
    \[
    Z[\rho, \bar{\rho}] = \int [d\bar{\psi}d\psi] \, e^{-\bar{\psi}_i A_{ij}\psi_j + \bar{\psi}_i \rho_i + \bar{\rho}_i \psi_i}    
    \]
    dove le sorgenti $\rho, \bar{\rho}$ sono anch'esse variabili di Grassmann anticommutanti con le $\psi, \bar{\psi}$.

    Svolgendo l'integrale si trova che
    \[
    Z[\rho, \bar{\rho}] = \det A e^{\bar{\rho}_i (A^{-1})_{ij}\rho_j}    
    \]
\end{example}  

\subsubsection{Regole di derivazione per le variabili di Grassmann}
\begin{enumerate}
    \item Se $f(\eta)$ non dipende da $\eta_i$ allora 
    \[
    \pdv{\eta_i}f(\eta) = f(\eta)\overleftarrow{\pdv{\eta_i}}=0    
    \]
    \item Se $\eta_i$ compare una sola volta in un certo prodotto di termini, allora 
    \begin{align*}
        \pdv{\eta_i}(\eta_i R) &= R \\
        (L \eta_i)\overleftarrow{\pdv{\eta_i}} = L 
    \end{align*}
\end{enumerate}
Utilizzando le nozioni di derivata destra e sinistra appena introdotte possiamo definire le funzioni di correlazioni fermioniche
\[
\left< \psi_{i_1} \dots \psi_{i_l} \bar{\psi}_{j_1} \dots \bar{\psi}_{j_l} \right> = \frac{1}{Z[0, 0]}\left[\pdv{\bar{\rho}_{i_1}} \dots \pdv{\bar{\rho}_{i_l}} Z[\rho, \bar{\rho}] \overleftarrow{\pdv{\rho_{j_1}}} \dots \overleftarrow{\pdv{\rho_{j_l}}}\right]_{\rho = \bar{\rho} = 0}  
\]
\begin{observation}
    Le funzioni di correlazione del tipo $\left< \psi_{i_1} \dots \psi_{i_l} \bar{\psi}_{j_1} \dots \bar{\psi}_{j_q} \right>$ con $l \neq q$ si annullano
\end{observation}



\section{Rottura spontanea di simmetria}

\begin{definition}[Rottura spontanea di simmetria (SSB)]
    La SSB si verifica quando l'azione ha una certa simmetria, ma la teoria ha una famiglia di vuoti degeneri che trasformano l'uno nell'altro sotto questa simmetria

\end{definition}
Consideriamo l'esempio di un ferromagnete, per cui

\[
\mathcal{H} = -J\sum_{\langle i,j \rangle} \mathbf{s}_i \cdot \mathbf{s}_j
\]

L'azione è invariante sotto rotazioni spaziali.

Sopra $T_c$ lo stato fondamentale è unico con magnetizzazione nulla; sotto $T_c$ la magnetizzazione è $\neq 0$, quindi possiamo rompere la simmetria ruotando attorno l'asse di magnetizzazione.

La rottura spontanea di simmetria è dovuta al fatto che il sistema sceglie un vuoto tra quelli disponibili.

\begin{observation}
Una caratteristica della SSB è l'esistenza di un parametro d'ordine che ha un valore d'aspettazione non nullo sul vuoto.
\end{observation}
Per il ferromagnete tale parametro è la magnetizzazione; per il caso di un potenziale a doppia buca
\[
V(\phi) = \frac{1}{2}\lambda^2(\phi^2 - \eta^2)^2    
\]
tale parametro è 
\[
\frac{<\phi>}{\eta} = \pm 1    
\]
In ogni caso il parametro d'ordine è una quantità \underline{non} invariante sotto la simmetria.
\\
Vediamo come questo si traduce nel caso dei gruppi di Lie.

Sia 
\[U = \exp(i\theta^a T^a)\]
Se il vuoto è invariante:
\[
    U\ket{0} = \ket{0} 
\]
quindi
\[T^a\ket{0} = 0 \quad \forall a\]
Altrimenti 
\[
\exists a \quad \textit{ t.c. } \quad T^a\ket{0} \neq 0    
\]
Nel caso di un ferromagnete, se la magnetizzazione è lungo z, allora $J_z\ket{0} = 0$, ma
\[
J_x \ket{0} \neq 0 \qquad J_y \ket{0} \neq 0    
\]
Nel caso di una doppia buca, considerando $\phi$ complesso \((V(\phi) = \frac{1}{2}\lambda^2 (|\phi|^2 - \eta^2)^2)\), abbiamo un set continuo di minimi
\[
\phi = \eta e^{i\alpha}    
\]
con 
\[
<|\phi|> = \eta \qquad <\alpha> = \textit{ arbitrario }  
\]
La lagrangiana 
\[
\mathcal{L} = \frac12\partial_\mu\phi^\dagger \partial^\mu\phi - V(\phi)   
\]
è invariante sotto $U(1)$, mentre la SSB fissa un minimo, cioè $<\alpha> = \alpha_0$ che possiamo prendere WLOG pari a 0. Quindi
\[
<\phi> = \eta    
\]
Studiamo ora lo spettro della teoria dopo la SSB, cioè sviluppando per piccole oscillazioni. Scriviamo

\[
\phi(x) = \eta + \frac{1}{\sqrt{2}}\left(\chi(x) + i\psi(x)\right)    
\]

con $\chi, \psi$ reali.

Il set di vuoti è allora un cerchio di raggio $\eta$, $\chi$ è una fluttuazione nella direzione ortogonale alla varietà dei vuoti, mentre $\psi$ è una fluttuazione tangenziale.

Quindi $\eta + i \psi$ è un altro vuoto per $\psi$ infinitesimo, quindi un piccolo spostamento lungo $\psi$ non costa energia. Possiamo perciò associare a $\psi$ un modo massless, analogamente associamo a $\chi$ un modo massivo.

Per vederlo algebricamente basta sostituire la nuova parametrizzazione di $\phi$ nella lagrangiana.

In generale vale il teorema di Goldstone che afferma la seguente proposizione.

\begin{thm}[Goldstone]
    Data una teoria Lorentz invariante, se una simmetria globale continua viene rotta spontaneamente, allora espandendo attorno al vuoto appare una particella massless per ogni generatore che rompe la simmetria ($T^a\ket{0} \neq 0$). Tale particella è detta Bosone di Goldstone.
\end{thm}

\begin{prop}
    La dimensione della varietà dei vuoti è pari al numero di generatori che rompono la simmetria.
\end{prop}

Introduciamo ora anche un campo di gauge $A_\mu$. Allora la lagrangiana sarà
\[
\mathcal{L} = - \frac{1}{4} F_{\mu\nu}F^{\mu\nu} + \frac12 D_\mu\phi^\dagger D^\mu\phi - V(\phi)
\]
con 
\[
D_\mu = \partial_\mu + iqA_\mu    
\]
Se parametrizziamo il campo come 
\[
\phi(x) = (\eta + \frac{1}{\sqrt{2}} \varphi(x))e^{i\alpha(x)}    
\]
possiamo porre $\alpha = 0$ tramite trasformazioni di gauge.
\\
Sostituendo infine nella lagrangiana otteniamo un campo di gauge massivo.
\\
L'equazione del moto è poi 
\[
\partial_\mu F^{\mu\nu} + m_A^2 A^\nu = 0    
\]
detta equazione di Proca. Contraendo poi con $\partial_\nu$ ed usando il fatto che \(\partial_\mu \partial_\nu F^{\mu\nu} = 0, m^2_A \neq 0\) otteniamo
\[
\partial_\mu A^\mu = 0    
\]
cioè
\[
(\square^2 + m^2_A)A^\nu = 0    
\]
L'equazione di Proca descrive allore un bosone di gauge massivo.
\\
Possiamo ora decomporre $A_\mu$ in $0 \oplus 1$. Espandendo $A_\mu$ in onde piane 
\[
\partial_\mu A^\mu = 0 \longrightarrow k_\mu \epsilon^\mu(k) = 0    
\]
che elimina la componente con vettore di polarizzazione $\epsilon_\mu(k) \sim k_\mu$ poichè per questa avremmo
\[
k_\mu \epsilon^\mu  \sim k^2 = m^2_A \neq 0   
\]
Nel sistema a riposo (che esiste poichè $m^2_A \neq 0$)
\[
k_\mu = m^2_A (1, 0, 0, 0)    
\]
quindi abbiamo eliminato la parte scalare di $A_\mu$ che è allora una particella a spin 1.

\begin{thm}[Goldstone]
    La rottura di una simmetria locale non produce allora bosoni di Goldstone, ma fa sì che i campi di gauge acquistino massa 
\end{thm}
In questo caso $\phi$ è detto \textit{campo di Higgs} ed il meccanismo che produce un bosone di gauge massivo è detto \textit{meccanismo di Higgs}.
\begin{observation}
    I gradi di libertà prima e dopo la SSB sono conservati
\end{observation}
La SSB è realizzata in natura da fenomeni quali la superconduttività.

\section{Gruppi di Lie}
Ricordiamo che $U(N)$ è il gruppo delle matrici unitarie $N\times N$, mentre $SU(N)$ è il gruppo delle matrici unitarie $N \times N$ a determinante 1.
\begin{observation}
    Una matrice $U(N)$ si può scrivere come una matrice $SU(N)$ con una fase, cioè
    \[
    U(N) = SU(N) \otimes U(1)    
    \]
\end{observation} 
Ricordiamo inoltre la notazione esponenziale, per cui una matrice U $N \times N$ si può scrivere come
\[
U = e^{A} \left( \equiv \sum_{n \neq 0} \frac{1}{n!} A^n \right)    
\]
\begin{prop}
    U unitaria $\Rightarrow A^\dagger = -A$, cioè A anti-hermitiana 
\end{prop}

\begin{proof}
    \[U^\dagger = e^{A^\dagger}, U^{-1} = e^{-A}\] U unitaria $\Rightarrow U^\dagger = U^{-1}$ \\
\end{proof}

Possiamo allora ridefinire 
\[U = e^{iH}\]
con H hermitiana ($H^\dagger = H$).
\\
Se poi $U \in SU(N)$ allora 
\[
\det U = \det e^A = e^{\tr A} = e^{i\tr H} = 1  
\]
cioè    
\[
    \tr A = \tr H = 0
\]
\\
La dimensione di $SU(N)$ è allora data dalla dimensione dello spazio delle matrici hermitiane a traccia nulla, cioè
\[
\dim SU(N) = N^2 - 1    
\]
Analogamente
\[
\dim U(N) = N^2    
\]
Quindi possiamo scrivere una generica matrice hermitiana a traccia nulla H come
\[
H = \sum_{a = 1}^{N^2 - 1} \theta_a F_a     
\]
con $\left\{ F_a  \right\}$ base dello spazio.

Allora
\[
U = \exp \left( i \theta_a F_a  \right) \equiv U(\theta)    
\]
dove i \(\theta_a \) sono detti parametri del gruppo e gli $F_a$ sono detti generatori del gruppo.

\subsection{Algebra di Lie}
Si può vedere che i generatori del gruppo di Lie soddisfano le regole di commutazione   
\[
    [F_a, F_b] = i c_{ab}^c F_c 
\]
dove gli $c_{ab}^c$ sono parametri reali e vengono chimati costanti di struttura del gruppo.

\begin{idea}
    \[
    U_1 U_2 = U_3 \in SU(N)    
    \]
    Svolgendo l'operazione si ottiene il risultato
\end{idea}
Con questa struttura si dice che i generatori del gruppo formano un'algebra di Lie (spazio vettoriale, spazio tangente)
\begin{prop}
    Per definizione 
    \[
    c_{ab}^c = -c_{ba}^c     
    \]
\end{prop}
Tramite un'opportuna scelta dei generatori possiamo rendere le costanti di struttura completamente antisimmetriche in tutti e 3 gli indici.

\begin{thm}
    Se i generatori sono hermitiani allora esiste una base per l'algebra di Lie 
    \[
    \tilde{F_a} = L_{ab} F_b     
    \]
    con $L_{ab}$ matrice $(N^2 - 1)\times(N^2 - 1)$ reale ed invertibile, tale che
    \[
    [\tilde{F_a}, \tilde{F_b}] = i \tilde{c}_{ab}^c \tilde{F_c}    
    \]
    con $\tilde{c}_{ab}^c$ antisimmetrici in $\{a, b, c\}$.
    
    Inoltre vale che
    \[
    \tr[\tilde{F_a}\tilde{F_b}] = \lambda \delta_{ab}, \qquad \lambda > 0     
    \]
\end{thm}

\begin{proof}
    Definiamo 
    \[
    g_{ab} \equiv \tr[F_a F_b]    
    \]
    che è una matrice reale, simmetrica e definita positiva, cioè una forma quadratica.

    Possiamo allora diagonalizzarla con matrici ortogonali
    \[
    F_a' = O_{ab} F_b    
    \]
    Allora 
    \[
    g_{ab}' = \tr [F_a' F_b'] = (O g O^t)_{ab} = \lambda_a \delta_{ab}. \qquad \lambda_a > 0     
    \]
    Posso poi riscalare ponendo
    \[
    \tilde{F_a} = \sqrt{\frac{\lambda}{\lambda_a}}F_a'    
    \]
    ed ottengo
    \[
    \tilde{g} = \lambda \mathbb{1}, \qquad \lambda > 0    
    \]
    Con questa scelta le costanti di struttura risultano completamente antisimmetriche
    \[
    [\tilde{F_a}, \tilde{F_b}] = i \tilde{c}_{ab}^c \tilde{F_c}    
    \]
    \[
    \Rightarrow \tilde{c}_{ab}^c = \frac{1}{i\lambda}\tr([\tilde{F_a}, \tilde{F_b}]\tilde{F_c})    
    \]
    usando $\tilde{g}_{ab} = \lambda \delta_{ab}$. Da cui la tesi usando la ciclicità della traccia.

\end{proof} 
Di seguito useremo la notazione per cui 
\[
    \tilde{c}_{ab}^c = f_{abc} \qquad \tilde{F_a} = F_a 
\]
e fisseremo $\lambda = \frac12$ per cui
\[
\tr(F_a F_b) = \frac12 \delta_{ab}    
\]

\begin{observation}[Rango]
    [\dots]
\end{observation}

\begin{definition}[Operatore di Casimir]
    Si dice operatore di Casimir un polinomio di generatori C t.c. commuta con tutti i generatori dell'algebra, cioè
    \[
    [C, F_a] = 0, \qquad \forall a     
    \]
\end{definition}

\begin{example}
    Per $SU(N)$
    \[
    F^{(2)} = \sum_{a = 1}^{N^2 - 1}F_a^2    
    \]
    è il Casimir quadratico.
\end{example}

\begin{definition}[Rango]
    Si dice rango dell'algebra di Lie il numero massimo di generatori che commutano tra loro, cioè sono simultaneamente diagonalizzabili.
\end{definition}

Il rango è inoltre la dimensione della sotto algebra di Cartan.

\begin{thm}[Racah]
    Il rango di $SU(N)$ è $N - 1$ ed è uguale al numero di operatori di Casimir funzionalmente indipendenti.
\end{thm}

\subsection{Rappresentazioni di $SU(N)$}

\begin{definition}
    Dato un gruppo G con elementi $g$, si dice rappresentazione del gruppo un omomorfismo tra gli elementi del gruppo e lo spazio delle applicazioni lineari $GL(\mathbb{C}^n)$ che preserva l'operazione di gruppo.
    
    Quindi R è una rappresentazione se 
    \[
    g_1 g_2 = g_3 \Rightarrow R(g_1) R(g_2) = R(g_3)    
    \]
\end{definition}
In questo caso $\mathbb{C}^n$ è lo spazio base della rappresentazione ed n è la dimensione.

Analogamente la rappresentazione di un'algebra di Lie è un omomorfismo tra i generatori e gli elementi di $GL(\mathbb{C}^n)$ che preserva la struttura dell'algebra, cioè
\[
[F_a, F_b] = i f_{abc}F_c \Rightarrow [F_a^{(r)}, F_b^{(r)}] = if_{abc}F_c^{(r)}    
\]
dove $F_a^{(r)}$ è il generatore nella rappresentazione r.

\begin{observation}
    Se la dimensione della rappresentazione è $d(r)$ allora le matrici sono $d(r) \times d(r)$
\end{observation}

Indicheremo con f la rappresentazione fondamentale.

Un esempio di rappresentazione che useremo in seguito è la rappresentazione aggiunta che ha dimensione $d(r) = N^2 - 1$ e 
\[
(F_a^{(agg)})_{bc} = -if_{abc}    
\]

\begin{prop}[Identità di Jacobi]
   \[ Cycl_{abc}([F_a, [F_b, F_c]]) = 0\]
\end{prop}

Dalla identità di Jacobi per i generatori se ne ricava una per le costanti di struttura del gruppo
\[
Cycl_{abc}(f_{aed}f_{bce}) = 0    
\]
che possiamo riscrivere in termini della rappresentazione aggiunta, cioè come
\[
[F_a^{(agg)}, F_b^{(agg)}] = i f_{abc}F_c^{(agg)}    
\]
che dimostra che l'aggiunta è effettivamente una rappresentazione.

\begin{observation}
    La rappresentazione aggiunta è ancora hermitiana
\end{observation}

\begin{definition}
Una rappresentazione si dice riducibile se $\exists V' \subseteq V = \mathbb{C}^n$ sottospazio vettoriale t.c. 
\[
R(g)V' \subseteq V' \qquad \forall g \in G    
\]
In termini di matrici questo implica che possiamo ridurre $R(g)$ a blocchi.
\end{definition}

\begin{example}
    Sia \( n' = \dim V' \), allora prendendo come base autovettori di $V'$ (più altro)
    \[
    R(g) = 
    \begin{bmatrix}
        R_1 & S \\
        0 & R_2
    \end{bmatrix}    
    \]
    dove $R_1$ è una matrice $n' \times n'$.
    Se accade che anche $S = 0$ allora R si dice completamente riducibile.
    In questo caso scriveremo
    \[
    R = R_1 \oplus R_2    
    \]
\end{example}
Se invece $\not\exists V'$ invariante allora R si dice irriducibile.

\begin{lemma}[Lemma di Schur]
    Sia R una rappresentazione irriducibile di un gruppo G e sia T un operatore lineare t.c.
    \[
    [T, R(g)] = \qquad \forall g \in G     
    \]
    allora
    \[
        T = \lambda\mathbb{1}\qquad \lambda \in \mathbb{C}
    \]
\end{lemma}
In particolare possiamo prendere come T i Casimir.

\begin{thm}
    La rappresentazione aggiunta di $SU(N)$ è irriducibile.
\end{thm}

\begin{thm}
    Presa r rappresentazione di $SU(N)$ con generatori $F_a^{(r)}$, si ha che
    \[
    g_{ab}^{(r)} = \tr[F_a^{(r)}F_b^{(r)}] =T_R^{(r)}\delta_{ab}    
    \]
    con $T_R^{(r)}$ detto indice di Dynkin della rappresentazione r.
\end{thm}

\begin{proof}
    Fissiamo una certa rappresentazione r ed omettiamo per semplicità l'apice che specifica la rappresentazione utilizzata.
    Definiamo 
    \[
    h_{abc} = \tr\left( [F_a, F_b]F_c \right)    
    \]
    Allora
    \[
    h_{abc} = i f_{abn} \tr[F_n F_c] = i f_{abn}g_{nc} = - h_{acb}    
    \]
    Quindi
    \[
    i f_{abn} g_{nc} = - i f_{acn}g_{nb}    
    \]
    ma g è simmetrico, quindi
    \[
    g_{cn}(- i f_{anb}) = -i f_{acn}g_{nb}   
    \]
    \[
    \Rightarrow g F_a^{(agg)} = F_a^{(agg)}g    
    \]
    cioè g commuta conn tutti i generatori dell'aggiunta, allora per il lemma di Schur g è un multiplo dell'identità.

\end{proof}

\begin{prop}
    Se r è irriducibile allora
    \[
    T_R^{(r)} = \frac{1}{N^2 - 1}d(r)C_F^{(r)}    
    \]
    con \(F_{(r)}^{(2)} = C_F^{(r)}\mathbb{1}_{d(r)\times d(r)}\)
\end{prop}

\begin{proof}
    Vedi esercizi.

\end{proof}


\begin{example}
    Per la rappresentazione fondamentale \(T_R = \frac12, d(f) = N\) quindi $C_F = \frac{N^2 - 1}{N}$.

    Nel caso della rappresentazione aggiunta $d(agg) = N^2 - 1$, quindi $T_R = C_F ( = N)$ (vedi esercizi).
\end{example}

\begin{prop}[Relazione di completezza per i generatori nella rappresentazione fondamentale]
    \[\sum_{a = 1}^{N^2 - 1}(F_a)_{ij}(F_a)_{kl} = \frac12 (\delta_{il}\delta_{kj} - \frac1N \delta_{ij}\delta_{kl})\]
\end{prop}

\begin{proof}
    Vedi esercizi

\end{proof}

\subsection{Rappresentazioni di $SU(2)$}
Vediamo ora il caso specifico delle rappresentazioni irriducibili di $SU(2)$.

Siano $T_a$ i generatori, allora
\[
[T_a, T_b] = i \epsilon_{abc}T_c    
\]
Riscriviamo l'algebra in termini di operatori di salita e discesa 
\[
T_\pm = T_1 \pm i T_2    
\]
In questo modo
\begin{align*}
[T_3, T_\pm] &= \pm T_\pm  \\  
[T_+, T_-] &= 2T_3    
\end{align*}

Prendiamo poi come set completo di operatori commutanti $T^{(2)}, T_3$ con 
\[
T^{(2)} = \sum_a T_a^2 = \frac12\{T_1, T_2\} + T_3^2    
\]
Gli autostati simultanei verrano indicati con 
\[
\ket{t, t_3}    
\]
e
\begin{align*}
    T^{(2)}\ket{t, t_3}  &= t(t + 1)\ket{t, t_3}  \\
    T_3\ket{t, t_3}  &= t_3\ket{t, t_3}  \\
    T_\pm\ket{t, t_3}  &= \sqrt{t(t + 1) - t_3(t_3 \pm 1)}\ket{t, t_3 \pm 1}  
\end{align*}
con 
\begin{align*}
    t &\in \{0, \frac12, 1, \frac32, 2, \dots\} \\
    t_3 &\in \{-t, -t + 1, \dots, t - 1, t\}
\end{align*}
I $2t + 1$ stati $\ket{t, t_3}$ per un dato t costituiscono la base dello spazio vettoriale su cui agisce la rappresentazione irriducibile di dimensione $2t + 1$.

Tale rappresentazione è detta di spin t e può essere rappresentata mediante diagramma di tipo peso

[Inserire diagrammi]

\subsection{Rappresentazioni irriducibili di $SU(3)$}
Riscriviamo come nel caso di $SU(2)$ l'algebra in termini di operatori di salita e discesa.

Osserviamo che $SU(3)$ ha rango 2, quindi ha 2 Casimir indipendenti, quello quadratico e quello cubico
\[
F^{(2)} = \sum_{a = 1}^{N^2 - 1}F_a^2, \quad G^{(3)} = \frac23 d_{abc}F_aF_bF_c     
\]
con $d_{abc}$ tensore completamente simmetrico di $SU(3)$.

Osserviamo poi che l'anticommutatore di due generatori nella fondamentale è ancora una matrice hermitiana $n\times n$, per cui, fissata la rappresentazione f 
\[
\{F_a, F_b\} = k_{ab} \mathbb{1}\footnote{Perchè non è a traccia nulla} + d_{abc}F_c    
\]
Tracciando l'anticommutatore si trova che 
\[
k_{ab} = \frac{\delta_{ab}}{N}.    
\]
Proiettando, invece, si trova che 
\[
d_{abc} = 2 \tr[\{F_a, F_b\}F_c]    
\]
\begin{observation}
    $d_{abc}$ per $SU(2)$ è zero, infatti non abbiamo un Casimir cubico per $SU(2)$.
\end{observation}
Un insieme completo di operatori commutanti è
\[
    F^{(2)}, G^{(3)}, T^{(2)}, Y, T_3
\]
dove
\[
T_3 = F_3, Y = \frac{2}{\sqrt{3}}F_8    
\]
Riscriviamo ora l'algebra.

Siano
\[
T_\pm = F_1 \pm iF_2, \quad U_\pm = F_4 \pm iF_5, \quad V_\pm = F_6 \pm iF_7    
\]
da cui
\begin{align*}
    [T_3, T_\pm] &= \pm T_\pm & [Y, T_\pm] &= 0 & [T_+, T_-] &= 2T_3 \\
    [T_3, U_\pm] &= \mp \frac12 U_\pm & [Y, U_\pm] &= U_\pm & [U_+, U_-] &=\frac32 Y - T_3 =2U_3 \\
    [T_3, V_\pm] &= \pm \frac12 V_\pm & [Y, V_\pm] &= \pm V_\pm & [V_+, V_-] &=\frac32 Y + T_3 =2V_3 \\
\end{align*}
Ogni riga definisce rispettivamente il sottogruppo chiuso di T-spin, U-spin e V-spin.

[Metodo grafico]

\subsection{Rappresentazione complessa coniugata}
Consideriamo una rappresentazione r 
\begin{prop}
   \[ U_1^{(r)}U_2^{(r)} = U_3^{(r)} \Rightarrow  U_1^{(r)^*}U_2^{(r)^*} = U_3^{(r)^*}\]
\end{prop}
Possiamo allora definire una rappresentazione $r^*$. In termini di generatori, se
\[
U^{(r)} = \exp(i \epsilon_a F_a^{(r)})\qquad \epsilon_a \in \mathbb{R}    
\]
allora
\[
U^{(r)^*} = \exp(-i \epsilon_a F_a^{(r)*}) = \exp(i \epsilon_a F_a^{(r^*)})
\]
quindi
\[
F_a^{(r^*)} = - F_a^{(r)*}    
\]

\begin{prop}
    \[
    [F_a^{(r)}, F_b^{(r)}] = i f_{abc}F_c^{(r)} \Rightarrow 
    [F_a^{(r*)}, F_b^{(r*)}] = i f_{abc}F_c^{(r*)}     
    \]
\end{prop}

Le due rappresentazioni sono distinte? $r, r^*$ non sono distinte se sono legate da una relazione di equivalenza:
\begin{definition}
    $r^*$ è equivalente ad $r$ se esiste $S$ matrice $d(r) \times d(r)$~\footnote{$d(r) = d(r^*)$} invertibile t.c.
    \[
    U^{(r^*)}(\epsilon) = S U^{(r)}(\epsilon)S^{-1}   \qquad \forall \epsilon
    \]
    ovvero
    \[
        F_a^{(r^*)}(\epsilon) = S F_a^{(r)}(\epsilon)S^{-1}   \qquad \forall a
    \]
    In tal caso r si dice reale.
\end{definition}

\begin{prop}
    Tutte le rappresentazioni irriducibili di $SU(2)$ sono reali.
\end{prop}

Ad esempio la rappresentazione fondamentale di $SU(2)$ è $F_a = \frac{\sigma_a}{2}$, allora
\[
F_a^{(f^*)} = -F_a^* = - \frac{\sigma_a}{2} = \sigma_2 F_a \sigma_2\qquad S = \sigma_2 = \sigma_2^{-1}
\]
\begin{prop}
    La rappresentazione aggiunta di $SU(N)$ è sempre reale
\end{prop}
\begin{proof}
    \[(F_a^{(agg)})_{bc} = -if_{abc}, \qquad f_{abc} \in \mathbb{R}\]
    \[
        (F_a^{(agg^*)})_{bc} = -(F_a^{(agg)})^*_{bc} = -(-i f_{abc}) = -if_{abc} = (F_a^{(agg)})_{bc} \qquad \forall a,b,c 
    \]

\end{proof}
\begin{thm}
    Se una data rappresentazione r di $SU(N)$ è reale, allora tutti i suoi generatori devono avere autovalori in coppie di segno opposto, cioè se $\lambda$ è autovalore anche $-\lambda$ deve esserlo.
\end{thm}
\begin{proof}
    $SU(N)$ è compatto quindi ogni sua rappresentazione è equivalente ad una rappresentazione unitaria, cioè una rappresentazione in cui i generatori sono hermitiani.

    WLOG consideriamo allora $F_a$ hermitiani: $F_a^{(r)^\dagger} = F_a^{(r)}$

    Poichè r reale per ipotesi $\exists S$ invertibile t.c.
    \[
        F_a^{(r^*)} = S F_a^{(r)}S^{-1} = - F_a^{(r)*}  
    \]
    Detto $\ket{\lambda}$ autovettore di $F_a^{(r)}$ con autovalore $\lambda$ si ha che $\lambda \in \mathbb{R}$ per hermitianità, inoltre
    \[
        F_a^{(r)^*}S\ket{\lambda} = - (S F_a^{(r)}S^{-1})S\ket{\lambda} = -S F_a^{(r)}\ket{\lambda} = -\lambda S\ket{\lambda} 
    \]
    Prendendone il complesso coniugato 
    \[
    F_a^{(r)}(S\ket{\lambda})^* = -\lambda(S\ket{\lambda})^*    
    \]
    che ci dice che anche $-\lambda$ è autovalore di $F_a^{(r)}$.
    
\end{proof}

\begin{observation}
    \[
    \ket{\lambda} \neq 0 \Rightarrow S\ket{\lambda} \neq 0   
    \]
    Poichè S è invertibile.
\end{observation}

\begin{cor}
    La rappresentazione fondamentale di $SU(3)$ non è reale, cioè 3 e $3^*$ non sono equivalenti.

\end{cor}

\begin{proof}
    \[
    F_8 = \frac{\lambda_8}{2} = \frac{1}{2\sqrt{3}}
    \begin{pmatrix}
        1 & & \\
        & 1 & \\
        & & -2
    \end{pmatrix}    
    \]
    che non ha autovalori in coppie di segno opposto.

\end{proof}

In generale per $SU(3)$ si ha che $(p,q)^* = (q, p)$.

\subsection{Metodo tensoriale}
Vediamo ora il metodo tensoriale per la classificazione delle rappresentazioni irriducibili di $SU(N)$. Siano $U$ le matrici nella rappresentazione fondamentale.

\begin{definition}[Vettori controvarianti e covarianti]
    Un vettore q si dice controvariante di $SU(N)$ se sotto una trasformazione del gruppo 
    \[
    q^i \to U^i_j q^j    
    \]
    Se poi prendiamo il complesso coniugato $\bar{q}_i \equiv (q^i)^*$, allora
    \[
    \bar{q}_i \to \bar{q}_i' = (U_{ij})^* \bar{q}_j = (U^\dagger)_i^j \bar{q}_j    
    \]
    In questo caso $\bar{q}$ si dice vettore covariante di $SU(N)$.
\end{definition}

Possiamo estendere tale definizione e definire un tensore di rango $(p, q)$ di $SU(N)$ \(T_{j_1 \dots j_q}^{i_1 \dots i_p}\) con 
\[    
    T_{j_1 \dots j_q}^{i_1 \dots i_p} \to T'\phantom{.}_{j_1 \dots j_q}^{i_1 \dots i_p} = U_{\alpha_1}^{i_1}\dots U_{\alpha_p}^{i_p} (U^\dagger)_{j_1}^{\beta_1} \dots (U^\dagger)_{j_q}^{\beta_q} T_{\beta_1 \dots \beta_q}^{\alpha_1 \dots \alpha_p}
\] 
Quindi un vettore controvariante trasforma secondo la rappresentazione fondamentale $N$ di $SU(N)$, un vettore covariante trasforma secondo $N^*$ ed un tensore $(p, q)$ trasforma come \[\underbrace{N \otimes \dots \otimes N}_{p}\otimes \underbrace{N^* \otimes \dots \otimes N^*}_{q} = N^p \otimes (N^*)^q\]

\subsection{Tensori di $SU(3)$}
Un tensore T di rango $(p, q)$ di $SU(3)$ si dice riducibile se mediante una contrazione con un tensore invariante ($\delta_{ij}, \epsilon_{ijk}$) del tipo
\begin{align*}
T_{j_1 \dots j_q}^{i_1 \dots i_p} \, \delta_{i_a}^{j_b}  &\to (p - 1, q - 1) \\
T_{j_1 \dots j_q}^{i_1 \dots i_p}\,  \epsilon^{i_{p + 1} j_b j_{b'}} &\to (p + 1, q - 2) \\
T_{j_1 \dots j_q}^{i_1 \dots i_p} \, \epsilon_{j_{q + 1} i_b i_{b'}} &\to (p - 2, q + 1)
\end{align*}
si ottiene un tensore $T'$ non nullo con rango più basso ($p' + q' < p + q$).

Se ciò non è possibile il tensore T è irriducibile.

\begin{prop}
    Il numero di componenti indipendenti di un tensore irriducibile di rango $(p, q)$ è 
    \[
    d(p, q) = \frac12 (p+1)(q+1)(p+q+2)    
    \]
\end{prop}

\begin{thm}
    Dato un generico tensore T di rango $(p, q)$ e dette $\phi_1, \dots, \phi_d$ le sue componenti linearmente indipendenti, sotto $SU(3)$ esse trasformeranno come
    \[
    \begin{pmatrix}
        \phi_1 \\
        \vdots \\
        \phi_d 
    \end{pmatrix}    
    \to 
    \begin{pmatrix}
        \phi_1' \\
        \vdots \\
        \phi_d' 
    \end{pmatrix} 
    = V
    \begin{pmatrix}
        \phi_1 \\
        \vdots \\
        \phi_d 
    \end{pmatrix}
    \]
    con V matrice $d \times d$ che costituirà una rappresentazione d-dimensionale di $SU(3)$.
\end{thm}

\begin{prop}
    T irriducibile $\Rightarrow$ V irriducibile.

    T riducibile $\Rightarrow$ V riducibile.
\end{prop}

\begin{prop}
    I tensori \(\delta_i^j, \epsilon_{ijk}, \epsilon^{ijk}\) sono tensori riducibili, ma essendo invarianti ad 1 componente rappresentano tutti la rappresentazione irriducibile $1 = (0, 0)$.
\end{prop}

\begin{prop}
    La decomposizione di un tensore riducibile come somma diretta di tensori irriducibili è una proprietà invariante sotto trasformazioni del gruppo.
\end{prop}

\begin{example}
    Vediamo a titolo d'esempio la decomposizione di un generico tensore $T^{ij}$ in parte simmetrica e parte antisimmetrica
    \begin{alignat*}{3}
        T^{ij} &= \frac12(T^{ij} + T^{ji}) &&+ \frac12 (T^{ij} - T^{ji}) \\
                &= S^{ij} &&+ A^{ij}      
    \end{alignat*}
    con
    \[
    S^{ij} =  \frac12(T^{ij} + T^{ji})   
    \]
    tensore irriducibile $(2, 0)$, 6 componenti.
    \[
    A^{ij} = \frac12 (T^{ij} - T^{ji}) 
    \]
    tensore asimmetrico con 3 componenti equivalente a 
    \[
    \bar{q}_k = \epsilon_{ijk} T^{ij} = \epsilon_{ijk} (S^{ij} + A^{ij}) = \epsilon_{ijk} A^{ij}    
    \]
    Infatti
    \[
    \epsilon^{ijk} \bar{q}_k = \epsilon^{ijk} \epsilon_{\alpha \beta k} T^{\alpha \beta} = (\delta_\alpha^i \delta_\beta^j - \delta_\beta^i \delta_\alpha^j)T^{\alpha \beta} = T^{ij} - T^{ji} = 2 A^{ij}     
    \]
    da cui
    \[
    A^{ij} = \frac12 \epsilon^{ijk} \bar{q}_k    
    \]
    Possiamo allora riscrivere 
    \[
    T^{ij} = S^{ij} + \frac12 \epsilon^{ijk}\bar{q}_k    
    \]
    che equivale alla decomposizione
    \[
    3 \otimes 3 = 6 \oplus 3^*    
    \]
\end{example}

\begin{observation}
    \[
    S'\phantom{.}^{ij} = U^i_\alpha U^j_\beta S^{\alpha \beta} \Rightarrow  S'\phantom{.}^{ij} = S'\phantom{.}^{ji}   
    \]
    \[
    A'\phantom{.}^{ij} = U^i_\alpha U^j_\beta A^{\alpha \beta} \Rightarrow  A'\phantom{.}^{ij} = -A'\phantom{.}^{ji}   
    \]
    quindi
    \[
    T'\phantom{.}^{ij} = U^i_\alpha U^j_\beta T^{\alpha \beta} = S'\phantom{.}^{ij} + A'\phantom{.}^{ij}   
    \]
    cioè viene ancora decomposto come $T^{ij}$ come asserito in una proposizione precedente.
\end{observation}

\begin{prop}
    Un tensore irriducibile $T_{j_1 \dots j_q}^{i_1 \dots i_p}$ sarà
    \begin{itemize}
        \item completamente simmetrico negli indici $\{i_1, \dots, i_p\}$
        \item completamente simmetrico negli indici $\{j_1, \dots, j_q\}$
        \item a traccia nulla
    \end{itemize}
\end{prop}

\begin{example}
    Un altro esempio di decomposizione utile di rappresentazione è quello che riguarda il tensore il tensore $T_j^i$
    \begin{alignat*}{3}
        T^i_j &= (T^i_j - \frac13 T_k^k \delta^i_j) &&+ \frac13 T_k^k \delta^i_j \\
        &= \tilde{T}^i_j &&+ T_k^k \delta^i_j 
    \end{alignat*}
    che corrisponde alla decomposizione 
    \[
    3 \otimes 3^* = 8 \oplus 1    
    \]
\end{example}

\begin{observation}
    Il tensore irriducibile $(q, p)$ trasforma come il complesso coniugato del tensore irriducibile $(p, q)$.
\end{observation}

\section{QED}
La lagrangiana della QED risulta essere
\[
    \lagr = \underbrace{\bar{\psi}(i\slashed{D}- m)\psi}_{\lagr_F} - \underbrace{\frac14 F_{\mu \nu}F^{\mu \nu}}_{\lagr_G}
\]
con
\[
D_\mu = \partial_\mu + iqA_\mu, \quad F_{\mu \nu} = \partial_\mu A_\nu - \partial_\nu A_\mu    
\]
Possiamo poi espandere $\lagr_F$ come
\begin{alignat*}{3}
    \lagr_F &= \lagr_0 &&+ \lagr_I \\
    &= \bar{\psi}(i\slashed{\partial}-m)\psi &&+ (-q\bar{\psi}\slashed{A}\psi)
\end{alignat*}
Per trasformazioni di gauge $U(1)$ globali
\begin{align*}
    \psi &\to e^{i\frac{q}{e}\theta}\psi \qquad \theta \text{ c.te} \\
    A_\mu &\to A_\mu
\end{align*}
sono invarianti $\lagr_0, \lagr_F, \lagr_G$, mentre per trasformazioni di gauge $U(1)$ locali
\begin{align*}
    \psi(x) &\to e^{i\frac{q}{e}\theta(x)}\psi(x)\\
    A_\mu(x) &\to A_\mu(x) - \frac{1}{e}\partial_\mu\theta(x)
\end{align*}
sono invarianti $\lagr_F, \lagr_G$.

In questo caso $D_\mu$ trasforma come $\psi$, infatti è detta derivata covariante
\begin{align*}
    D_\mu\psi(x) \to D'_\mu\psi'(x) &= \left(\partial_\mu + iqA'_\mu(x)\right)e^{i\frac{q}{e}\theta(x)}\psi(x) \\
    &= e^{i\frac{q}{e}\theta(x)}\left(\partial_\mu + i\frac{q}{e}\partial_\mu\theta + iqA_\mu - i\frac{q}{e}\partial_\mu\theta\right)\psi(x) \\
    &= e^{i\frac{q}{e}\theta(x)}D_\mu\psi(x)    
\end{align*}

\begin{observation}[$F_{\mu \nu}$ come commutatore delle derivate covarianti]
    \begin{align*}
        [D_\mu, D_\nu]\psi &= [\partial_\mu + iqA_\mu, \partial_\nu + iqA_\nu]\psi = iq([\partial_\mu, A_\nu] + [A_\mu, \partial_\nu])\psi \\
        &= iq \left(\partial_\mu(A_\nu \psi) - A_\nu\partial_\mu\psi + A_\mu\partial_\nu\psi - \partial_\nu(A_\mu\psi)\right) \\
        &= iq (\partial_\mu A_\nu \psi + A_\nu \partial_\mu \psi - A_\nu\partial_\mu\psi - (\mu \leftrightarrow \nu )) \\
        &= iq (\partial_\mu A_\nu - \partial_\nu A_\mu)\psi \\
        &= iq F_{\mu \nu}\psi
    \end{align*}
In questo caso il commutatore tra i campi di gauge è nullo perchè gli $A_\mu$ sono numeri.
\end{observation}

\subsection{Quantizzazione teorie di gauge}
Se consideriamo una lagrangiana del tipo
\[
\lagr = -\frac14 F_{\mu\nu}F^{\mu\nu}    
\]
con $F_{\mu\nu} = \partial_\mu A_\nu - \partial_\nu A_\mu$, il momento coniugato ad $A_\mu$ è 
\[
\pi_\mu = \pdv{\lagr}{\dot{A}_\mu} = -F_{0\mu}    
\]
Dalle relazioni di commutazione dovremmo avere 
\[
[A^{\mu}(x), \pi_\nu(y)] = i\delta^\mu_\nu \delta^4(x - y)    
\]
tuttavia $\pi_0 = 0$.

Per risolvere il problema dobbiamo fissare una gauge.

\begin{example}[Propagatore del fotone]
    Vediamo fisicamente come si traduce questo problema e per farlo calcoliamo il propagatore del fotone o meglio mostriamo che non si può definire.
    \begin{align*}
    S[A] = \int \lagr[A] &= -\frac14 \int (\partial_\mu A_\nu - \partial_\nu A_\mu)(\partial^\mu A^\nu - \partial^\nu A^\mu)      \\
    &= -\frac12 \int [(\partial_\nu A_\mu)(\partial^\nu A^\mu) - (\partial_\nu A_\mu)(\partial^\mu A^\nu)]   \\
    &= \frac12 \int A_\mu [g^{\mu \nu} \partial^2 - \partial^\mu \partial^\nu]A_\nu \\
    &= \frac12 \int \frac{d^4 k}{(2\pi)^4} \tilde{A}_\mu(k)[-k^2 g^{\mu\nu} + k^\mu k^\nu]\tilde{A}_\nu(k)
    \end{align*}
    L'operatore è singolare, infatti ha dei modi zero
    \[
        [g^{\mu \nu} \partial^2 - \partial^\mu \partial^\nu] \partial_\nu \theta = 0 \forall \theta
    \]
    $\partial_\nu\theta$ è essenzialmente una trasformazione di gauge del campo nullo $A_\nu = 0$.

    Non possiamo allora definire il propagatore del fotone.
\end{example}

\section{Metodo di Faddeev-Popov}
\label{section:FP}
    Per risolvere il problema si può usare il metodo di Faddeev-Popov. Consideriamo la funzione di partizione 
    \[
    Z = \int DA\,e^{i \int d^4x\, [-\frac14 F_{\mu\nu}^aF^{\mu\nu, a}]}    
    \]
\begin{observation}[Invarianza misura]
    Osserviamo subito che sia la lagrangiana, sia la misura sono invarianti di gauge, in particolare per la misura si vede che sotto una trasformazione di gauge infinitesima si ha
    \[
    A'\phantom{.}_\mu^a = A_\mu^a - f_{abc} \theta^b(x) A_\mu^c(x) - \frac{1}{g}\partial_\mu\theta^a(x)    
    \]
    per cui
    \[
    DA' = DA \det N , \qquad N = \fdv{A'\phantom{.}_\mu^a(x)}{A_\nu^b(y)}   
    \]
    Allora
    \[
    N = \delta^4(x - y) \delta_\mu^\nu\, [\delta_{ab} - f_{adc}\theta^d(x)\delta_{cb}] = \delta^4(x - y) \delta_\mu^\nu \, [\delta_{ab} + f_{acd}\theta^d(x)\delta_{cb}]    
    \]
    Utilizzando adesso che
    \[
    \det(\id + L) = 1 + \tr L + \dots    
    \]
    e che
    \[
    f_{aab} = 0    
    \]
    per asimmetria, si ha che
    \[
    \det N = 1    
    \]
    quindi la misura è invariante di gauge.
\end{observation}

\begin{definition}[Orbita di gauge]
    Dato un campo di gauge $A_\mu$ definiamo orbita di gauge l'insieme dei campi $A_\mu^U$ che si ottengono da $A_\mu$ mediante trasformazione di gauge
    \[
    A_\mu^U = U A_\mu U^{-1} + \frac{i}{g}(\partial_\mu U)U^{-1}    
    \]         
\end{definition}

\begin{definition}[Gauge fixing]
    Definiamo condizione di gauge o gauge fixing una condizione del tipo
    \[
    G^\mu A_\mu^{U, a}(x) = B^a(x)    
    \]        
    che per una data orbita di gauge ammette un'unica soluzione $U$.
    [Inserire immagine gauge fixing!!]
\end{definition}
\begin{example}
    Un esempio di gauge fixing è dato dalla scelta
    \[
    G^\mu = \partial^\mu \qquad B^a = 0    
    \]
\end{example}

Per introdurre il metodo di FP vediamo anzitutto il primo trucco di FP, cioè
\[
1 = \Delta_G[A] \int DU\, \delta(G^\mu A_\mu^U - B)    
\]
dove $DU$ è la misura invariante di Haar.

\begin{definition}[Misura left di Haar]
    Dato un gruppo G con elementi g si dice che $d_L g$ è una misura left-invariante se 
    \[
        \int f(g_0 \cdot g)\,d_L g = \int f(g)\,d_L g \qquad \forall g_0 \in G
    \]
    cioè se 
    \[
    d_L(g_0 \cdot g) = d_L(g) \qquad \forall g_0 \in G    
    \]
\end{definition}
Analogamente si definisce una misura right-invariante di Haar.

Nel caso di nostro interesse le due misure coincidono e si parla di misura di Haar.

Notiamo ora che $\Delta_G[A]$ è invariante di gauge, infatti
\begin{align*}
    \Delta_G[A^{U'}]^{-1} &= \int DU\,\delta(G^\mu(A_\mu^{U'})^U - B) \\
    &= \int DU\,\delta(G^\mu A_\mu^{U'U} - B)    \\
    &= \int DU\, \delta(G^\mu A_\mu^U - B) \\
    &= \Delta_G[A]^{-1} \qquad\qquad\forall U' \in SU(N_c)
\end{align*}
Utilizziamo questa relazione nella funzione di partizione 
\[
Z = \int DA\, e^{iS[A]}\Delta_G[A]\int DU\,\delta(G^\mu A_\mu^U - B)  
\]
Possiamo ora cambiare variabile di integrazione
\[
A_\mu \to A_\mu^{U'}    
\]
che equivale ad usare l'invarianza di gauge e porre $U' = U^{-1}$, da cui
\begin{align*}
    Z &= \int DA\, e^{iS[A]}\Delta_G[A]\delta(G^\mu A_\mu - B) \int DU \\
    &= \tilde{Z}\int DU
\end{align*}
Abbiamo in questo modo isolato la divergenza della funzione di partizione. 

Calcoliamo ora un'espressione esplicita per $\Delta_G[A]$
\begin{align*}
    \Delta_G[A]^{-1} &= \int DU\, \delta(G^\mu A_\mu^U - B) \\
    &= \int D\theta\, J[\theta] \delta(G^\mu A_\mu^{U(\theta)} - B) \\
    &= \frac{J[\theta]}{\det M_G}
\end{align*}
con 
\[
(M_G(x, y))_{ab} = \eval{\fdv{(G^\mu A_\mu^{U(\theta)a})}{\theta_b(y)}}_{\theta = \bar{\theta}}    
\]
dove $\bar{\theta}$ è l'unica soluzione di 
\[
    G^\mu A_\mu^{U(\theta)a} = B^a
\]
\begin{observation}
    Abbiamo alcune libertà sui termini di FP. Possiamo infatti porre $\theta = 0$ e $J[\theta] = 1$ così da ottenere
    \[
    \tilde{Z} = \int DA\, \det M_G \delta(G^\mu A_\mu - B)e^{iS[A]}   
    \]
    Essendo $B^a(x)$ arbitrario e siccome $\tilde{Z}$ non dipende da B, possiamo dare una dinamica al campo in forma di termine lagrangiano aggiuntivo
    \[
    \lagr \to \lagr - \frac{1}{2\alpha}(B^a(x))^2 = \lagr - \frac{1}{2\alpha}(G^\mu A_\mu^a)^2      
    \]
    Il termine aggiuntivo fissa la gauge.
\end{observation}


\section{QCD}
\subsection{Modello a quark di Gell-mann \& Zweig}
Si osserva che i mesoni più leggeri, con \(J^p = 0^-, J^p = 1^-\), occupano entrambi ottetti e singoletti di \(SU(3)\), mentre i barioni più leggeri, con \(J^p = \frac12^+, J^p=\frac32^+\), occupano ottetti e decupletti di \(SU(3)\).
Tenendo inoltre presente che 
\begin{align*}
    3 \otimes 3^* &= 8 \oplus 1 \\
    3 \otimes 3 \otimes 3 &= 1 \oplus 8 \oplus 8 \oplus 10
\end{align*}
G. \& Z. proposero l'idea che mesoni e barioni fossero stati legati di costituenti fondamentali detti quarks e delle loro antiparticelle dette antiquarks, che occupano rispettivamente la rappresentazione \(3\) e \(3^*\) di \(SU(3)\).

In questo schema i mesoni risultano stati legati quark-antiquark con numero barionico \(B = 0\); se si assume poi che i quark abbiano spin \(\frac12\) i mesoni risulteranno avere spin intero, in particolare, negli stati di momento angolare orbitale più basso \((L = 0)\) si avrà \(J = 0\) o \(J = 1\).

I barioni, invece, risultano stati legati di 3 quarks (gli antibarioni di 3 antiquarks) con numero barionico \(B = 1\, (B = -1)\), per cui assegniamo \(B = \frac13\) ai quarks e \(B = -\frac13\) agli antiquarks. Lo spin \(\frac12\) dei quarks implica poi spin semi-intero per i barioni: \(J = \frac12, J = \frac23\) per gli stati con \(L = 0\).

Ci sono poi 6 flavour
\[
\begin{pmatrix}
u \\ d    
\end{pmatrix}
,
\begin{pmatrix}
    c \\ s
\end{pmatrix}
,
\begin{pmatrix}
    t \\ b 
\end{pmatrix}
\]
La prima riga ha carica elettrica \(\frac23\), la seconda \(-\frac13\).

\subsubsection{Paradossi modello a quark}
\begin{enumerate}
    \item Perchè non si osservano quark isolati nè stati legati adronici come $qq, \dots$? \label{conf}
    \item Ci sono problemi nel costruire le funzioni d'onda barioniche: lo stato \(\Delta^{++}\), ad esempio, è costituito da 3 quarks di tipo \(u\) in onda s, quindi con i 3 spin allineati
    \[
    \ket{\Delta^{++}, J = \frac32} = \ket{u_{(1)}\uparrow, u_{(2)}\uparrow, u_{(3)}\uparrow}_{l = 0} 
    \] 
    Ma i 3 \(u\) sono fermioni identici, per cui avremmo una violazione della Fermi-Dirac. \label{l}
\end{enumerate}
Per risolvere \ref{l}. è stato proposto l'esistenza di un grado di libertà nascosto detto di colore a cui associamo il gruppo $SU(3)$

[Inserire schema flavour-colour]

Quindi 
\[
    \ket{\Delta^{++}, J = \frac32} = \frac{1}{\sqrt{3}}\epsilon_{ijk} \ket{u_{(1)}^i\uparrow, u_{(2)}^j\uparrow, u_{(3)}^k\uparrow}_{l = 0} 
\] 
con \(i = 1, 2, 3\) indice di colore.

Tale stato risulta essere un singoletto di colore 
\[
u^i \to u'\phantom{.}^i = U^i_j u^j,\qquad U^i_j \in SU(3)_{\text{colour}}    
\]
Inoltre per una data configurazione di flavour non si osservano partner mesonici con diversi numeri quantici di colore, quindi anche la funzione d'onda mesonica costituisce un singoletto di colore.
\[
\ket{M} = \frac{1}{\sqrt{3}}\ket{q^i \bar{q}_i}    
\]
Invece stati del tipo $qq, \bar{q}\bar{q}, \dots$ non possono essere resi singoletti di colore in quanto nella decomposizione delle rappresentazioni non compare la 1
\begin{align*}
    qq: 3_c \otimes 3_c &= \bar{3}^*_c \oplus 6_c \\
    \bar{q}\bar{q}: \bar{3}^*_c \otimes \bar{3}^*_c &= 3_c \oplus \bar{6}^*_c
\end{align*}
Per risolvere \ref{conf}. si guarda agli esperimenti da cui segue il postulato del confinamento: \textit{tutti gli stati adronici e le corrispondenti osservabili fisiche sono singoletti di colore}.

\subsubsection{Jet adronici}
[Vedi note]

\subsection{Derivata covariante}
Vogliamo rendere $SU(3)_{\text{colore}}$ il gruppo di gauge con cui costruire la derivata covariante, tuttavia lavoriamo nel caso generico di una teoria di gauge non abeliana $SU(N_c)$, dove $N_c$ sono i campi di colore.
Abbiamo allora che
\[
\lagr_0 = \sum_{i=1}^{N_c}\bar{\psi}_i(i\slashed{\partial} - m)\psi_i = \bar{\psi}(i\slashed{\partial} - m)\psi    
\]
con $\psi$ vettore dei campi di colore.

Consideriamo ora una trasformazione di gauge non abeliana
\begin{align*}
    \psi &\to \psi' = U\psi \\
    \bar{\psi} &\to \bar{\psi'} = \bar{\psi}U^\dagger
\end{align*}
con $U \in SU(N_c)$. Per trasformazioni locali $U = U(x)$ e 
\[
\partial_\mu \psi' = \partial_\mu(U\psi) = U\partial_\mu\psi + \partial_\mu U \psi    
\]
quindi
\[
i\bar{\psi'}\slashed{\partial}\psi' = i\bar{\psi}\slashed{\partial}\psi + i\bar{\psi}(U^\dagger\slashed{\partial}U)\psi    
\]
Possiamo ora pensare di aggiungere un termine di accoppiamento minimale 
\[
\bar{\psi}\slashed{A}\psi    
\]
con $A_\mu = A_\mu^a T^a$, $A_\mu^a$ campi di gauge.

La lagrangiana fermionica trasformata risulta allora essere
\[
\lagr_F' = \bar{\psi'}(i\slashed{\partial} - g\slashed{A}' - m)\psi' = \bar{\psi}(i\slashed{\partial} - m)\psi + \bar{\psi}(iU^\dagger\slashed{\partial}U - gU^\dagger\slashed{A}'U))\psi    
\]
Se ora imponiamo che $\lagr_F' = \lagr_F$ si ha che
\[
-g\slashed{A} = iU^\dagger\slashed{\partial}U - gU\dagger\slashed{A}'U    
\]
cioè
\[
    gU^\dagger\slashed{A}'U = iU^\dagger\slashed{\partial}U + g\slashed{A}
\]
che in componenti si scrive
\[
A_\mu' = UA_\mu U^\dagger + \frac{i}{g}(\partial_\mu U)U^\dagger    
\]
Possiamo allora definire una derivata covariante
\[
D_\mu \psi = (\partial_\mu + igA_\mu)\psi    
\]
che sotto gauge trasforma appunto in maniera covariante, cioè
\[
D_\mu \psi \to UD_\mu\psi    
\]
In analogia a quanto fatto per la QED svolgiamo il commutatore delle derivate covarianti per trovare il tensore energia-impulso della teoria
\[
[D_\mu, D_\nu]\psi = ig(\partial_\mu A_\nu - \partial_\nu A_\mu + ig[A_\mu, A_\nu])\psi \equiv ig F_{\mu\nu}\psi    
\]
che in componenti si scrive 
\begin{align*}
    F_{\mu\nu} &= F_{\mu\nu}^a T^a = (\partial_\mu A_\nu^a - \partial_\nu A_\mu^a)T^a + ig[T^b, T^c]A_\mu^b A_\nu^c \\
&= (\partial_\mu A_\nu^a - \partial_\nu A_\mu^a - gf_{abc}A_\mu^b A_\nu^c)T^a 
\end{align*}

Consideriamo ora una trasformazione di gauge infinitesima $U(x) = \exp(i \theta^a T^a) \simeq \id + i\theta^a T^a $, in questo caso 
\[
A_\mu' = UA_\mu U^\dagger + \frac{i}{g}(\partial_\mu U)U^\dagger \simeq (A_\mu^a - f_{abc}\theta^b A_\mu^c - \frac{1}{g}\partial_\mu\theta^a)T^a    
\] 
cioè
\[
\delta A_\mu^a  = A_\mu' - A_\mu = -f_{abc}\theta^b A_\mu^c\phantom{.}\footnote{Indica che $A_\mu^a$ trasforma secondo la rappresentazione aggiunta di $SU(N_c)$} - \frac{1}{g}\partial_\mu\theta^a   
\]
Il campo, invece, trasforma secondo la rappresentazione fondamentale di $SU(N_c)$
\[
\psi' = U\psi \simeq (1 + i\theta^a T^a)\psi    
\]
\[
\Rightarrow \delta\psi = \psi' - \psi = i\theta^a T^a \psi    
\]
Infine l'intensità di campo trasforma come
\begin{align*}
    F'_{\mu \nu} &= U F_{\mu\nu} U^\dagger = U T^a U^\dagger F_{\mu\nu}^a \\
    &\simeq (1 + i\theta^b T^b)T^a (1 - i\theta^c T^c)F_{\mu\nu}^a \\
    &= (T^a + i \theta^b [T^b, T^a])F_{\mu\nu}^a \\
    &= (T^a - i\theta^b if_{abc}T^c)F_{\mu\nu}^a \\
    &= T^a (F_{\mu\nu}^a + \theta^b f_{cba}F_{\mu\nu}^c) 
\end{align*}
cioè
\[
\delta F_{\mu\nu}^a = F'^a_{\mu\nu} - F_{\mu\nu}^a = -f_{abc}\theta^b F_{\mu\nu}^c     
\]
quindi il tensore dei campi trasforma secondo la rappresentazione aggiunta di $SU(N_c)$ anche per $\theta$ non costanti.

Risulta allora che \(F_{\mu\nu}^a F^{\mu\nu,a}\) è invariante di gauge, infatti
\[
\delta(F_{\mu\nu}^a F^{\mu\nu,a}) = 2 F^{\mu\nu,a} \delta F_{\mu\nu}^a = -2\underbrace{f_{abc}}_{\text{asym}}\theta^b \underbrace{F^{\mu\nu,a}F_{\mu\nu}^c}_{\text{sym } a \leftrightarrow c} = 0     
\]
Inoltre usando 
\[
\tr [T_a T_b] = \frac12 \delta_{ab}    
\]
si ha che
\[
    F_{\mu\nu}^a F^{\mu\nu,a} = 2 \tr[F_{\mu\nu}F^{\mu\nu}]    
\]
che è una quantità gauge invariante per ciclicità della traccia.

\subsubsection{Trasporto parallelo}
\begin{definition}[Trasporto parallelo]
    Il Trasporto parallelo di un vettore $\psi \in V_x$ lungo una curva \(C_{y \leftarrow x}\) da $x$ ad $y$, definita come
    \[
        C_{y \leftarrow x}:\, z^\mu(s), \, s \in [0,1] \text{ t.c. } z^\mu(0) = x, z^\mu(1) = y
    \]
    è un operatore (o matrice) \(W (C_{y \leftarrow x}) \in SU(N_c)\) che opera una trasformazione da $V_x$ a $V_y$
    \[
    W(C_{y \leftarrow x}):V_x \to V_y    
    \]
    Il vettore \(\tilde{\psi} = W(C_{y \leftarrow x})\psi(x) \in V_y\) viene definito come il vettore $\psi(x)$ trasportato parallelamente.

    I trasporti paralleli devono inoltre soddisfare le seguenti proprietà:
    \begin{enumerate}
        \item \(W(\emptyset) = \id\), con $\emptyset$ curva di lunghezza nulla
        \item \(W(C_2 \circ C_1) = W(C_2)W(C_1)\) \label{comp}
        \item \(W(-C) = W(C)^{-1}\)
        \item Sotto gauge locale \label{tr par}
        \begin{align*}
            \psi(x) &\to U(x)\psi(x) \\
            \psi(y) &\to U(y)\psi(y) \\
        \end{align*}
        W trasforma come
        \[
            W(C_{y \leftarrow x}) \to U(y)W(C_{y \leftarrow x})U(x)^{-1}    
            \]
        \end{enumerate}
\end{definition}
Consideriamo ora un percorso infinitesimo
\[
C \equiv C_{x + dx \leftarrow x}    
\]
Al prim'ordine in $dx$ il trasporto parallelo è
\[
W(C) \simeq \exp(-igA_\mu(x)dx^\mu) \simeq 1 - igA_\mu dx^\mu 
\]
con \(A_\mu(x) \in \mathfrak{su}(N_c)\) algebra del gruppo.

Determiniamo ora come trasforma il campo di gauge $A_\mu$.
\begin{align*}
W'(C) &= \id -  igA'_\mu dx^\mu = U(x + dx)W(C)U(x)^{-1} = U(x + dx)(\id - igA_\mu dx^\mu)U(x)^{-1} \\
&= U(x + dx)U(x)^{-1} - igU(x + dx)A_\mu U(x)^{-1}dx^{\mu}    
\end{align*}
Espandendo ora al prim'ordine $U(x + dx)$ si ottiene
\begin{align*}
    W'(C) &= (U(x) + \partial_\mu U(x)dx^\mu)U(x)^{-1} - igU(x)A_\mu U(x)^{-1}dx^\mu \\
    &= \id - ig [UA_\mu U^{-1} + \frac{i}{g} (\partial_\mu U)U^{-1}]dx^\mu
\end{align*}
quindi
\[
A'_\mu(x) = U(x)A_\mu(x)U(x)^{-1} + \frac{i}{g}(\partial_\mu U(x))U(x)^{-1}    
\]

\begin{observation}
    Vale anche il viceversa, cioè se si assume la trasformazione dei campi si ottiene la \ref{tr par}.
\end{observation}

Possiamo ora definire il differenziale covariante di $\psi(x)$
\begin{align*}
    D\psi(x) &= W(C_{x \leftarrow x + dx})\psi(x + dx) - \psi(x) \\
    &= W(C_{x + dx \leftarrow x})^{-1} \psi(x + dx) - \psi(x) \\
    &= (1 + igA_\mu dx^\mu)\psi(x + dx) - \psi(x) \\
    &= (\psi(x + dx) - \psi(x)) + +igA_\mu(x)\psi(x + dx)dx^\mu \\
    &= (\partial_\mu \psi(x) + igA_\mu (x)\psi(x))dx^\mu = D_\mu \psi(x)dx^\mu
\end{align*}
Per costruzione il differenziale covariante trasforma in maniera covariante.

Assegnato ora il campo di gauge $A_\mu$ mostriamo come ricostruire il trasporto parallelo per una curva qualsiasi. Sia
\[
C_\tau: z^\mu(s),\,s\in [0, \tau]
\]
e facciamo variare il punto finale, cioè consideriamo $C_{\tau + d\tau}$. Poichè
\[
z^\mu(\tau + d\tau) = z^\mu(\tau) + \dv{z^\mu}{\tau} d\tau    
\]
possiamo scrivere usando \ref{comp}.
\[
    W(C_{\tau + d\tau}) = W(C_{z + \dv{z}{\tau}\leftarrow z})W(C_\tau) = (\id - igA_\mu(z(\tau)) \dv{z^\mu(\tau)}{\tau}d\tau)W(C_\tau)
\]
Quindi
\[
W(C_{\tau + d\tau}) - W(C_\tau) = -igA_\mu \dv{z^\mu}{\tau}d\tau W(C_\tau)    
\]
\[
\Rightarrow \dv{W(C_\tau)}{\tau} = -igA_\mu\dv{z^\mu}{\tau}W(C_\tau)    
\]
Usando la chain rule si ha che
\[
\dv{W(C_\tau)}{\tau} = \dv{z^\mu}{\tau} \pdv{W(C_\tau)}{z^\mu}    
\]
da cui otteniamo
\[
\dv{z^\mu}{\tau}\left[\pdv{z^\mu} + igA_\mu\right]W(C_\tau)=0    
\]
cioè
\[
    \dv{z^\mu}{\tau}D_\mu W(C_\tau) = 0
\]

La soluzione dell'equazione con condizione al contorno $W(C_0) = \id$ è data dalla formula di Dyson
\[
W(C_\tau) = P \exp\left(-ig \int_{C_\tau}A_\mu(z)dz^\mu \right)    
\]
con $P$ path-ordering.

\subsection{QCD perturbativa}
Consideriamo una lagrangiana del tipo
\[
\lagr = \lagr_G + \lagr_F + \lagr_{GF}    
\]
dove i singoli termini sono
\begin{align*}
    \lagr_G &= -\frac14 F_{\mu\nu}^aF^{\mu\nu, a} \\
    \lagr_F &= \sum_{f=1}^{N_f} \bar{\psi}_f (i\slashed{D} - m_f)\psi_f \\
    \lagr_{GF} &= -\frac{1}{2\alpha}(\partial^\mu A_\mu^a)^2  
\end{align*}
Riprendendo quanto fatto in \ref{section:FP} possiamo scrivere
\[
\tilde{Z} = \int DA D\bar{\psi}D\psi \, \det M_G e^{iS}    
\]
Ricordando che per le variabili di Grassmann vale
\[
\int d\bar{\eta}d\eta e^{-\bar{\eta}^t A \eta} = \det A    
\]
Possiamo scrivere $\det M_G$ come un integrale nelle variabili di Grassmann. A meno di una costante, infatti, abbiamo che
\begin{align*}
    \det M_G &\propto \int D\bar{c}Dc\, \exp\left(-\int d^4x d^4y\,\bar{c}_a(x)(-ig M_G(x, y))_{ab}c_b(y) \right) \\
    &= \int D\bar{c}Dc\, \exp\left(-\int d^4x d^4y\,\bar{c}_a(x)(-\partial^\mu)(D_\mu^{(agg)})_{ab}c_b(y) \right) \\
    &= \int D\bar{c}Dc\, e^{i\int d^4x\, \lagr_{FP}}
\end{align*}
I $c, \bar{c}$ sono detti campi di ghost e sono scalari benchè anticommutanti. Non violano però il teorema di spin-statistica in quanto non esistono come eccitazioni.

Quindi il termine aggiuntivo di Faddeev-Popov può essere visto come un termine aggiuntivo nella lagrangiana
\end{document}